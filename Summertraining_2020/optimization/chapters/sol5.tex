\renewcommand{\theequation}{\theenumi}
\begin{enumerate}[label=\arabic*.,ref=\thesubsection.\theenumi]
\numberwithin{equation}{enumi}
\item  given that $\to$
\begin{align}
Z &= 5x + 3y
\\
\text{subjected to}
\\
3x + 5y &\leq 15
\\
5x + 2y &\leq 10
\\
x  &\geq 0
\\
y &\geq 0
\end{align}
Comparing above equation to the gernelised form$\to$ 
\begin{align}
maxZ &= c^t \vec{x}
\\
\text{subjected to}
\\
A\vec{x} &\preceq b
\\ 
\text{we can find $\to$ }
\\
c &= \myvec{5\\ 3}
\\
A &= \myvec{3 & 5 \\ 5 & 2\\1 &0 \\ 0 & 1}
\\
b &= \myvec{15\\10\\ 0\\ 0}
\end{align}\\
Solution for the above equations can be find from the python code of Lenear programing 
\begin{lstlisting}
./optimization/codes/lp5.eps
\end{lstlisting}	
From the codes we get that the maximum value of the equation will be 12.37 at the point $\myvec{1.05 & 2.37}$.

\begin{figure}[!ht]
	\centering
	\includegraphics[width=\columnwidth]{./figures/lp5.eps}
	\caption{ lp5}
	\label{fig:lp5}
	Pythone codes for the above figure can be get from
	\begin{lstlisting}
	./optimization/figures/lp5.eps
	\end{lstlisting}	
\end{figure}
\end{enumerate}