\documentclass{beamer}
\usepackage{multicol}
\setlength{\columnsep}{1cm}
\usepackage{subfigure}
\usepackage{graphicx}
\usepackage{caption}
\usetheme{Antibes}
%maths package
\usepackage{amsmath}
\usepackage{xfrac}
\usepackage{nicefrac}

\usepackage{color}                                             \usepackage{array}                                            
\usepackage{longtable}                                        
\usepackage{calc}                                             
\usepackage{multirow}                                           \usepackage{hhline}                                              \usepackage{ifthen}    


% for numbered equation
\usepackage[utf8]{inputenc}

\usepackage{tcolorbox}

\title{\textbf{Solution For School Geometry Problems}}
\author{Yogesh Choudhary}


\begin{document}
	
	\def\inputGnumericTable{}
	\frame{
			\titlepage
		  }
	  
	 \frame{
	  		\frametitle{Question}
	  		\begin{enumerate}
	  		\item 
	  		\textbf{ Two sides AB and BC and median AM of one triangle ABC are respectively equal to sides PQ and QR and median PN of $\Delta$ PQR.Show that:}
	  				
	  			\begin{enumerate}
	  				\item$\Delta$ ABM $\cong$ $\Delta$ PQN  
	  				\item$\Delta$ ABC $\cong$ $\Delta$ PQR
	  			\end{enumerate}
	  	   \end{enumerate}	
  	      }	
        
       \begin{frame}
       		\frametitle{Construction}
       	\begin{enumerate}
       		\item 
       	
       		\begin{multicols}{2}
       				We have the values of all three sides of the triangle ABC and PQR so to construct  a triangle we need all three coordinates of A,B and C.
       			
       			
       			
       		\begin{table}[h!]
       			\begin{center}
       				\caption{table for the output.}
       				\label{tab:table1}
       				\input{inp.tex}
       			\end{center}
       		\end{table}
       	
       		\begin{align}
       		x = \frac{\left(a^2 + c^2 - b^2\right)}{2*a}
       		\\
       		y=\sqrt{c^2 - x^2}
       		\end{align}
       		
       		coordinates of A $\to$
       		
       		\begin{align}
       		x_A = x       		
       		\\
       		y_A = y
       		\end{align}	
       		\end{multicols}			
       	\end{enumerate}	
       \end{frame}
	  	
	  	\begin{frame}
	  		\begin{multicols}{2}
	  				coordinates of B $\to$
	  			\begin{align}
	  			x_B = 0       		
	  			\\
	  			y_B = 0
	  			\end{align}
	  			
	  			coordinates of C $\to$
	  			\begin{align}
	  			x_C = a      		
	  			\\
	  			y_C = 0
	  			\end{align}
	  			
	  			coordinates of M $\to$
	  			\begin{align}
	  			x_M = \frac{a}{2}       		
	  			\\
	  			y_M = 0
	  			\end{align}
	  			
	  			
	  			\begin{table}[h!]
	  				\begin{center}
	  					\caption{table for the output.}
	  					\label{tab:table1}
	  					\input{derived.tex}			
	  				\end{center}
	  			\end{table}
	  			
	  		\end{multicols}
	  	\end{frame}
  	
	    \begin{frame}    	
			\frametitle{Figures}
	    		 Let assume we have two triangles as follows$\to$\\
	    		 
	    	\begin{figure}[!htb]
	    		\begin{minipage}{0.48\textwidth}
	    			\centering
	    			\includegraphics[width=2.0in]{./codes/triangle.pdf}
	    			\caption{Triangle ABC}
	    			\label{fig:triangle}
	    		\end{minipage}
	    		\hfill
	    		\begin{minipage}{0.48\textwidth}
	    			\centering
	    			\includegraphics[width=1.0in]{./figures/congurentpicabc2.pdf}
	    			\caption{Triangle PQR}
	    			\label{fig:triangle2}
	    		\end{minipage}	
	    	\end{figure}
	    \end{frame}   
        
       \begin{frame}
        \begin{figure}	
        
            	\frametitle{Codes}
        	\item[]
        	\begin{tcolorbox}[colback=blue!5,colframe=blue!40!black,title=latex codes for figures a and b ]
        		
        	  ./figures/congurentpicabc2.pdf \\	
        	  ./figures/congurentpicabc.pdf \\
        	   ./figures/triangle.pdf
        	\end{tcolorbox}
        	
        	\begin{tcolorbox}[colback=red!5,colframe=blue!40!black,title=python codes for figures a and b ]
        		./codes/congurenttriangle.py \\	
        		../codes/congurenttriangle2.py	 
        	\end{tcolorbox}
        \end{figure}
        \end{frame}
     
     
        \begin{frame}
        	\frametitle{Ans.1}
        	\begin{multicols}{2}
        	given that $\to$\\
        	
        	\begin{align}
        		AB = PQ\\
        		AM = PN\\
        		BC = QR	
        	\end{align}
        	
        	from equation $\left(13\right)$...
        	
        	\begin{align}
        		\frac{BC}{2} = \frac{QR}{2} \\
        		BM = QN
        	\end{align}
        %	\end{multicols}
       	%\end{frame}
       
       %\begin{frame}
       		from fig $\left[1\right]$ and $\left[2\right]$ ...
       	
       		\begin{align}
       			AB = PQ\\
       			AM = PN\\
       			BM = QN\\
       			\implies  \Delta ABM \cong \Delta PQN
       		\end{align}
       		\end{multicols}
       \end{frame}
       	
       	\begin{frame}
       		\frametitle{Ans.2}
       		\begin{multicols}{2}
       		given that $\to$\\
       		\begin{align}
       			AM = PN
       		\end{align}
       		
       		from equation $\left(13\right)$...
       		
       		\begin{align}
       			\frac{BC}{2} = \frac{QR}{2} \\
       			MC = NR
       		\end{align}	
       	%\end{frame}
       
    % \begin{frame}
     		from equation $\left(19\right)$...
     	\begin{align}
     		\Delta ABM \cong \Delta PQN 
     		\\
     		\implies \angle AMB = \angle PNQ
     		\\
     		180 - \angle AMB = 180 -  \angle PNQ
     		\\
     		\angle AMC = \angle PNR
     	\end{align}
     	\end{multicols}
     \end{frame}
     \begin{frame}
     	\begin{multicols}{2}
     	from equation $\left(10\right)$,$\left(12\right)$ and $\left(16\right)$...
     	\begin{align}
     		AM = PN
     		\\
     		MC = NR
     		\\
     		\angle AMC = \angle PNR
     		\\
     		\implies  \Delta AMC \cong \Delta PNR
     		\\
     		\implies AC = PR
     	\end{align}
     	
    % \end{frame}
 
 %	\begin{frame}
 		from equation $\left(11\right)$,$\left(13\right)$ and $\left(31\right)$...
 	
 		\begin{align}
 			AB = PQ\\
 			BC = QR\\
 			AC = QR\\
 			\implies  \Delta ABC \cong \Delta PQR
 		\end{align}
 		\end{multicols}
 	\end{frame}
\end{document}