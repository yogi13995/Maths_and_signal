%\renewcommand{\theequation}{\theenumi}
\begin{enumerate}[label=\thesection.\arabic*.,ref=\thesection.\theenumi]
%\numberwithin{equation}{enumi}
\numberwithin{equation}{subsection}
\item We need draw the a circume circle for that we have all three sides of triangle.In order to construct the circumecircle first of all we will find all three coordinates of the triangle using the sides.

\item Values of all three sides of the triangle are as given in the table .


%\renewcommand{\thefigure}{\theenumi.\arabic{figure}}
\begin{figure}[!ht]
\centering
\resizebox{\columnwidth}{!}{\input{./figures/C_circle.tex}}
\caption{ circumecircle generated by latex}
\label{fig:C_circle}	
\end{figure}
%
%
%\renewcommand{\thefigure}{\theenumi}
%


%
\begin{table}[ht!]
\centering
%\begin{tabular}{ |p{3cm}|p{3cm}|  }
%\hline
% \multicolumn{2}{|c|}{Initial Input Values.} \\
%\hline
%a & 4\\
%\hline
%b & 3\\
%\hline
%$\phase{(ACB)$ & $90^{\circ}$ \\
%\hline
%\end{tabular}
\input{./tables/inp.tex}
\caption{To construct circumecircle}
\label{table:table1}	
\end{table}

\item Finding out  the coordinates of the various points in Fig. \ref{fig:C_circle}
\\
% 
%$\triangle ABC$ are 
\begin{align}
x_1 = \frac{\left(a^2 + c^2 - b^2\right)}{2*a}
\\
y_1 = \sqrt{c^2 - x_1^2}
\\
x_2 = \frac{\left(a^2 + b^2 - c^2\right)}{2*a}
\\
y_2 = \sqrt{b^2 - x_2^2}
\\
x_3 = a
\\
y_3 = \frac{a}{x_2}*{y_2}
\\
\myvec{\vec{A}} &= \myvec{x_1 \\ y_1}
\\
\myvec{\vec{B}} &= \myvec{0 \\ 0}
\end{align}

\begin{align}
\myvec{\vec{C}} &= \myvec{a\\ 0}
\\
\myvec{\vec{D}} &= \myvec{x_2 \\ y_2}
\\
\myvec{\vec{E}} &= \myvec{x_3 \\ y_3}
\end{align}
\item Finding the circumecentre $\to$
\\
let assume that circumecentre of the triangle ABC is $\vec{O}$
\begin{align}
\norm{\vec{A}-\vec{O}} = \norm{\vec{B}-\vec{O}} = \norm{\vec{C}-\vec{O}} = \norm{\vec{D}-\vec{O}}
\\
\norm{\vec{A}-\vec{O}}^2 - \norm{\vec{B}-\vec{O}}^2 = 0
\end{align}
Which can be simplified as

\begin{align}
\myvec{\vec{A}-\vec{B}}^T \vec{O} = \frac{(\norm{A}^2 -\norm{B}^2)}{2}
\end{align}
Similarly,
\begin{align}
\myvec{\vec{B}-\vec{C}}^T \vec{O} = \frac{(\norm{B}^2 -\norm{C}^2)}{2}
\end{align}
can be combined to form the matrix equation 

\begin{align}
\vec{N}^T = \vec{c}
\\
\vec{O} = \vec{N}^{-T} \vec{c}
\end{align}
Where
\begin{align}
\vec{N} = \myvec{\vec{A} -\vec{B} & \vec{B} -\vec{C}}
\\
\vec{c} = \frac{1}{2}\myvec{\norm{A}^2 -\norm{B}^2  \norm{B}^2 - \norm{C}^2}
\end{align}
\item  Finding $\vec{R}$ of circumecircle

\begin{align}
\vec{R} = \norm{\vec{B}- \vec{O}}
\end{align}
 

The values are listed in 
%\item List the  derived values.
%\label{const:table2}
%\\
%\solution See  
Table. \ref{table:table2} 
\begin{table}[ht!]
\centering
\begin{tabular}{ |p{3cm}|p{3cm}|  }
\hline
 \multicolumn{2}{|c|}{Derived Values.} \\
\hline
$\vec{O}$ & $$\begin{pmatrix}2.5\\1.7008\end{pmatrix}$$\\						
\hline
\end{tabular}
\caption{circumecentre of the triangle}
\label{table:table2}
\end{table}


%
\item Drawing  Fig. \ref{fig:c_circle}.	
\\
 The  following Python code generates Fig. \ref{fig:c_circle}
%
\begin{lstlisting}
codes/c_circle.py
\end{lstlisting}
\begin{figure}[!ht]
\centering
\includegraphics[width=\columnwidth]{./figures/c_circle.eps}
\caption{circumecircle generated using python}
\label{fig:c_circle}
\end{figure}

%
and the equivalent latex-tikz code generating Fig.2.1  is 
\begin{lstlisting}
figs/C_circle.tex
\end{lstlisting}
%
The above latex code can be compiled as a standalone document as
\begin{lstlisting}
figs/C_circle_slone.tex
\end{lstlisting}

%

%

%
%

\end{enumerate}

