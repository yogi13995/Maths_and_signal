%\renewcommand{\theequation}{\theenumi}
\begin{enumerate}[label=\thesection.\arabic*.,ref=\thesection.\theenumi]
%\numberwithin{equation}{enumi}

%
\item To prove that all four points lie on the circumefarence of the circle first of all we will draw a circume circle for a triangle ABC.

\item Finding the circumecentre $\to$
\\
let assume that circumecentre of the triangle ABC is $\vec{O}$
\begin{align}
\norm{\vec{A}-\vec{O}} = \norm{\vec{B}-\vec{O}} = \norm{\vec{C}-\vec{O}}
\\
\norm{\vec{A}-\vec{O}}^2 - \norm{\vec{B}-\vec{O}}^2 = 0
\end{align}
Which can be simplified as

\begin{align}
\myvec{\vec{A}-\vec{B}}^T \vec{O} = \frac{(\norm{A}^2 -\norm{B}^2)}{2}
\end{align}
Similarly,
\begin{align}
\myvec{\vec{B}-\vec{C}}^T \vec{O} = \frac{(\norm{B}^2 -\norm{C}^2)}{2}
\end{align}
can be combined to form the matrix equation 

\begin{align}
\vec{N}^T = \vec{c}
\\
\vec{O} = \vec{N}^{-T} \vec{c}
\\
\vec{O} = \myvec{2.5\\1.7}
\end{align}
Where
\begin{align}
\vec{N} = \myvec{\vec{A} -\vec{B} & \vec{B} -\vec{C}}
\\
\vec{c} = \frac{1}{2}\myvec{\norm{A}^2 -\norm{B}^2  \norm{B}^2 - \norm{C}^2}
\end{align}
\item  Finding $\vec{R}$ of circumecircle
\\
area of triangle of ABC $\to$
\begin{align}
\frac{1}{2}ab\sin{C}& = \frac{abc}{4R}
\\
\implies \vec{R} &= \frac{abc}{4s(\sqrt{\left(s-a\right)\left(s-b\right)\left(s-c\right)})}
\\
\vec{R}& = 3.023
\end{align}

\item For point D to be on the circumeference of the circume circle it should satisfie the circle equation 

\begin{align}
\norm{\vec{D} - \vec{O}}   &=  \norm{\vec{R}}
\\
\norm{\myvec{4.5 - 2.5 \\3.9686 -1.7}} &= 3.023
\\
\newline
\norm{\myvec{2\\2.26}} = 3.023
\end{align}

\item thus the point D setisfies the  circle equation of the circumecircle of triangle ABC and we can say that all four points lies on the circle.




\end{enumerate}
