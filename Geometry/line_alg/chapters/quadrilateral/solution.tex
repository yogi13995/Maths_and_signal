\renewcommand{\theequation}{\theenumi}
\begin{enumerate}[label=\arabic*.,ref=\thesubsection.\theenumi]
\numberwithin{equation}{enumi}

\item let assume that the vertices of the rhombus are$\vec P$, $\vec Q$,$\vec R$ and $\vec S$ respectively as shown in fig(2.2.1).
\\ finding out the $\vec SP$ and $\vec QP$...
\\
\begin{align}
	\vec {SP} = \vec P - \vec S = \myvec{3 +2\\0+1}
	\\ = \myvec{5\\1}
	\\
	\vec {PQ} = \vec Q - \vec P = \myvec{4-3\\5-0}
	\\ = \myvec{1\\5}
\end{align}

\begin{figure}[!ht]
	\centering
	\includegraphics[width=\columnwidth]{./figures/quad.eps}
	\caption{quadrilateral }
	\label{fig:quadrilateral}
\end{figure}
\begin{lstlisting}
codes/quad.py
\end{lstlisting}

$\vec S$ Area of the rhombus can be calculated as follows 
\\
\begin{align}
	\norm{\Delta} =abs\norm{ \mathbf{\vec{SP}} \times \mathbf{\vec{PQ}}}
	\\
	\norm{\Delta} = abs\norm{ \mathbf{\myvec{5\\1}} \times \mathbf{\myvec{1\\5}}}
	\\
	\norm{\Delta} = 5*5 - 1*1
	\\
	\norm{\Delta} = 24
\end{align}

\end{enumerate}